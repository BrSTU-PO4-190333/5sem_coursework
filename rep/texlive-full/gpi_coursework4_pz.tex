\documentclass[12pt, a4paper, simple]{eskdtext}

\usepackage{env}
\usepackage{_sty/gpi_lstlisting}
\usepackage{hyperref}
\usepackage{_sty/gpi_table_of_content}

% Код
\def \gpiDocTypeNum {81}
\def \gpiDocVer {00}
\def \gpiCode {\gpiLetterI\gpieLetterII\gpiLetterIII.\gpiStudentGroupName\gpiStudentGroupNum.\gpiStudentCard.0\gpiDocNum~\gpiDocTypeNum~\gpiDocVer}

% Графа 1 (наименование изделия/документа)
\ESKDcolumnI {\ESKDfontIII
    \gpiTopic \\
    Пояснительная записка
}

% Графа 2 (обозначение документа)
\ESKDsignature {\gpiCode}

% Графа 4 (литералы)
\ESKDcolumnIVfI {\gpiLetterI}
\ESKDcolumnIVfII {\gpieLetterII}
\ESKDcolumnIVfIII {\gpiLetterIII}

% Графа 9 (наименование или различительный индекс предприятия) задает команда
\ESKDcolumnIX {\gpiDepartment}

% Графа 11 (фамилии лиц, подписывающих документ) задают команды
\ESKDcolumnXIfI {\gpiStudentSurname}
\ESKDcolumnXIfII {\gpiTeacherSurname}
\ESKDcolumnXIfV {\gpiTeacherSurname}

\begin{document}
    % Титульный лист
    \input{_tex/gpi_pz_titlePage.tex}
    
    % Содержание
    \tableofcontents
    \paragraph{ПРИЛОЖЕНИЕ A ТЕКСТ ПРОГРАММЫ}
    \paragraph{ПРИЛОЖЕНИЕ Б СХЕМА АЛГОРИТМА}
    \newpage

    % Введение
    \newpage

\section*{ВВЕДЕНИЕ} % Секция без номера
\addcontentsline{toc}{section}{ВВЕДЕНИЕ} % Добавить в содержание

В современном мире человеку приходится сталкиваться с огромными массивами однородной информации. Эту информацию необходимо упорядочить каким-либо образом, обработать однотипными методами и в результате получить сводные данные или разыскать в массе конкретную информацию. Этой цели служат базы данных.

База данных — это организованная структура, предназначенная для хранения, изменения и обработки взаимосвязанной информации, преимущественно больших объемов.

Использование баз данных имеет ряд преимуществ:

\begin{enumerate}
    \item компактность – информация хранится в БД, нет необходимости хранить многотомные бумажные картотеки;
    \item скорость – скорость обработки информации (поиск, внесение изменений) компьютером намного выше ручной обработки;
    \item низкие трудозатраты – нет необходимости в утомительной ручной работе над данными;
    \item применимость – всегда доступна свежая информация.
\end{enumerate}

На сегодняшний день применение баз данных приобрело весьма важное значение для многих организаций, которые для упрощения своей работы применяют компьютерные технологии.

Основная цель работы - создание веб приложения, предоставляющего пользователю инструменты для работы с массивом структурированных данных,
содержащем в себе информацию о товарах.

    \newpage
\end{document}

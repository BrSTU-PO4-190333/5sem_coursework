\documentclass[12pt, a4paper, simple]{eskdtext}

\usepackage{env}
\usepackage{_sty/gpi_lstlisting}
\usepackage{hyperref}
\usepackage{_sty/gpi_table_of_content}
\usepackage{_sty/gpi_figures}

% Код
\def \gpiDocTypeNum {81}
\def \gpiDocVer {00}
\def \gpiCode {\gpiLetterI\gpieLetterII\gpiLetterIII.\gpiStudentGroupName\gpiStudentGroupNum.\gpiStudentCard.0\gpiDocNum~\gpiDocTypeNum~\gpiDocVer}

% Графа 1 (наименование изделия/документа)
\ESKDcolumnI {\ESKDfontIII
    \gpiTopic \\
    Пояснительная записка
}

% Графа 2 (обозначение документа)
\ESKDsignature {\gpiCode}

% Графа 4 (литералы)
\ESKDcolumnIVfI {\gpiLetterI}
\ESKDcolumnIVfII {\gpieLetterII}
\ESKDcolumnIVfIII {\gpiLetterIII}

% Графа 9 (наименование или различительный индекс предприятия) задает команда
\ESKDcolumnIX {\gpiDepartment}

% Графа 11 (фамилии лиц, подписывающих документ) задают команды
\ESKDcolumnXIfI {\gpiStudentSurname}
\ESKDcolumnXIfII {\gpiTeacherSurname}
\ESKDcolumnXIfV {\gpiTeacherSurname}

\begin{document}
    \input{_tex/gpi_pz_titlePage.tex}               % Титульный лист
    \tableofcontents                                
\paragraph{ПРИЛОЖЕНИЕ A ТЕКСТ ПРОГРАММЫ}
\paragraph{ПРИЛОЖЕНИЕ Б СХЕМА АЛГОРИТМА}

\newpage
          % Содержание
    \newpage

\section*{ВВЕДЕНИЕ} % Секция без номера
\addcontentsline{toc}{section}{ВВЕДЕНИЕ} % Добавить в содержание

В современном мире человеку приходится сталкиваться с огромными массивами однородной информации. Эту информацию необходимо упорядочить каким-либо образом, обработать однотипными методами и в результате получить сводные данные или разыскать в массе конкретную информацию. Этой цели служат базы данных.

База данных — это организованная структура, предназначенная для хранения, изменения и обработки взаимосвязанной информации, преимущественно больших объемов.

Использование баз данных имеет ряд преимуществ:

\begin{enumerate}
    \item компактность – информация хранится в БД, нет необходимости хранить многотомные бумажные картотеки;
    \item скорость – скорость обработки информации (поиск, внесение изменений) компьютером намного выше ручной обработки;
    \item низкие трудозатраты – нет необходимости в утомительной ручной работе над данными;
    \item применимость – всегда доступна свежая информация.
\end{enumerate}

На сегодняшний день применение баз данных приобрело весьма важное значение для многих организаций, которые для упрощения своей работы применяют компьютерные технологии.

Основная цель работы - создание веб приложения, предоставляющего пользователю инструменты для работы с массивом структурированных данных,
содержащем в себе информацию о товарах.
                   % Введение
    \newpage

\section{ПОСТАНОВКА ЗАДАЧИ}

% = = = = =

\subsection{Перечень функций}

Достижение цели курсовой работы предполагает необходимость создания приложения
для работы с локальной базой данных на тему <<Товары>> с использованием пользовательского
интерфейса и решения следующих конкретных задач:

\begin{enumerate}
    \item добавление товара в базу данных;
    \item вывод товаров из базы данных;
    \item удаление товара из базы данных;
    \item изменение товара в базу данных;
    \item сохранение товаров из базы данных в файл JSON;
    \item сохранение товаров из базы данных в файл CSV;
    \item открытие файла JSON и добавление товаров в базу данных.
\end{enumerate}

% = = = = =

\subsection{Требования пользователей}

Пользовательские требования - описание на естественном языке (плюс поясняющие диаграммы) функций,
выполняемых системой, и ограничений, накладываемых на неё.

Источники: Пользователь

Документ: Пользовательские требования / требования к ПО.

Ответственный: Системный аналитик.

Эти требования должны определять только внешнее поведение системы,
избегая по возможности определения структурных характеристик системы.
Пользовательские требования должны быть написаны естественным языком с использованием
простых таблиц, а также наглядных и понятных диаграмм.

Требования пользователя к информационной системе:

Обязательные:

\begin{enumerate}
    \item должна быть реализована функция ввода данных в информационную систему;
    \item должна работать функция удаления записи из информационной системы.
\end{enumerate}

Желательные:

\begin{enumerate}
    \item информационная система должна выводить отчеты на печать.
\end{enumerate}

\newpage
        % Постановка задачи
    \newpage

\section*{Список использованных источников}
\addcontentsline{toc}{section}{Список использованных источников}

\begin{enumerate}
    \item Получение GET и POST запросов на Node.js \\
    \url{https://youtu.be/YMJDUHUccvA}

    \item Подключение к базе данных MySQL в Node.js \\
    \url{https://youtu.be/YhuozY-qplI}

    \item Модули Node.js, require \\
    \url{https://youtu.be/1PkarXC-9TQ}
\end{enumerate}
              % СПИСОК ИСПОЛЬЗОВАННЫХ ИСТОЧНИКОВ
\end{document}

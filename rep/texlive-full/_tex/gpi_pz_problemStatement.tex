\newpage

\section{ПОСТАНОВКА ЗАДАЧИ}

% = = = = =

\subsection{Перечень функций}

Достижение цели курсовой работы предполагает необходимость создания приложения
для работы с локальной базой данных на тему <<Товары>> с использованием пользовательского
интерфейса и решения следующих конкретных задач:

\begin{enumerate}
    \item добавление товара в базу данных;
    \item вывод товаров из базы данных;
    \item удаление товара из базы данных;
    \item изменение товара в базу данных;
    \item сохранение товаров из базы данных в файл JSON;
    \item сохранение товаров из базы данных в файл CSV;
    \item открытие файла JSON и добавление товаров в базу данных.
\end{enumerate}

% = = = = =

\subsection{Требования пользователей}

Пользовательские требования - описание на естественном языке (плюс поясняющие диаграммы) функций,
выполняемых системой, и ограничений, накладываемых на неё.

Источники: Пользователь

Документ: Пользовательские требования / требования к ПО.

Ответственный: Системный аналитик.

Эти требования должны определять только внешнее поведение системы,
избегая по возможности определения структурных характеристик системы.
Пользовательские требования должны быть написаны естественным языком с использованием
простых таблиц, а также наглядных и понятных диаграмм.

Требования пользователя к информационной системе:

Обязательные:

\begin{enumerate}
    \item должна быть реализована функция ввода данных в информационную систему;
    \item должна работать функция удаления записи из информационной системы.
\end{enumerate}

Желательные:

\begin{enumerate}
    \item информационная система должна выводить отчеты на печать.
\end{enumerate}

\newpage

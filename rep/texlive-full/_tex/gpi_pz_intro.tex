\newpage

\section*{ВВЕДЕНИЕ} % Секция без номера
\addcontentsline{toc}{section}{ВВЕДЕНИЕ} % Добавить в содержание

В современном мире человеку приходится сталкиваться с огромными массивами однородной информации. Эту информацию необходимо упорядочить каким-либо образом, обработать однотипными методами и в результате получить сводные данные или разыскать в массе конкретную информацию. Этой цели служат базы данных.

База данных — это организованная структура, предназначенная для хранения, изменения и обработки взаимосвязанной информации, преимущественно больших объемов.

Использование баз данных имеет ряд преимуществ:

\begin{enumerate}
    \item компактность – информация хранится в БД, нет необходимости хранить многотомные бумажные картотеки;
    \item скорость – скорость обработки информации (поиск, внесение изменений) компьютером намного выше ручной обработки;
    \item низкие трудозатраты – нет необходимости в утомительной ручной работе над данными;
    \item применимость – всегда доступна свежая информация.
\end{enumerate}

На сегодняшний день применение баз данных приобрело весьма важное значение для многих организаций, которые для упрощения своей работы применяют компьютерные технологии.

Основная цель работы - создание веб приложения, предоставляющего пользователю инструменты для работы с массивом структурированных данных,
содержащем в себе информацию о товарах.

\newpage

\section*{Введение} % Секция без номера
\addcontentsline{toc}{section}{Введение} % Добавить в содержание

В современном мире человеку приходится сталкиваться с огромными массивами однородной информации.
Эту информацию необходимо упорядочить каким-либо образом, обработать однотипными методами
и в результате получить сводные данные или разыскать в массе конкретную информацию.
Этой цели служат базы данных.

База данных — это организованная структура, предназначенная для хранения,
изменения и обработки взаимосвязанной информации, преимущественно больших объемов.

Использование баз данных имеет ряд преимуществ:

\begin{enumerate}
    \item компактность - информация хранится в БД, нет необходимости хранить
    многотомные бумажные картотеки;
    \item скорость - скорость обработки информации (поиск, внесение изменений)
    компьютером намного выше ручной обработки;
    \item низкие трудозатраты - нет необходимости в утомительной ручной работе над данными;
    \item применимость - всегда доступна свежая информация.
\end{enumerate}

На сегодняшний день применение баз данных приобрело весьма важное значение для многих организаций,
которые для упрощения своей работы применяют компьютерные технологии.

Основная цель работы - создание веб приложения, предоставляющего пользователю инструменты для работы
с массивом структурированных данных, содержащем в себе информацию о товарах.

В современном мире программисту приходится сталкиваться с огромным количеством информации,
которую необходимо сохранить.
Хранение информации - это ее запись во вспомогательные запоминающие устройства на различных носителях
для последующего использования.
Хранение является одной из основных операций, осуществляемых над информацией,
и главным способом обеспечения ее доступности в течение определенного промежутка времени.

Информационная система (ИС) - система, предназначенная для хранения, поиска и обработки информации,
и соответствующие организационные ресурсы (человеческие, технические, финансовые и т. д.),
которые обеспечивают и распространяют информацию.

ИС предназначена для своевременного обеспечения надлежащих людей надлежащей информацией,
то есть для удовлетворения конкретных информационных потребностей в рамках определённой
предметной области, при этом результатом функционирования информационных систем является
информационная продукция - документы, информационные массивы, базы данных и информационные услуги.

Первоначально для хранения информации на ЭВМ применялись локальные массивы (или файлы),
при этом для каждой из решаемых функциональных задач создавались собственные файлы исходной
и результатной информации. Это приводило к значительному дублированию данных,
за счёт чего использовалось больше памяти вычислительной машины,
а также усложнялось обновление хранимой информации.

База данных представляет собой определенным образом структурированную совокупность данных,
совместно хранящихся и обрабатывающихся в соответствии с некоторыми правилами.
Как правило, БД моделирует некоторую предметную область или ее фрагмент.
Очень часто в качестве постоянного хранилища информации баз данных выступают файлы.

Немаловажной является и взаимосвязь информации в базе данных: изменение одной строчки
может привести к значительным изменениям других строк.
Работать с данными таким образом гораздо проще и быстрее,
чем если бы изменения касались только одного места в базе данных.

Помимо основной функции - хранения и систематизации огромного количества информации - они
позволяют быстро обрабатывать клиентские запросы и выдавать актуальную информацию.

На сегодняшний день базы данных занимают одно из первых мест для многих организаций,
которые для упрощения своей работы применяют компьютерные технологии.

Результатом разрабатываемой программы должно являться приложение,
позволяющее пользователю взаимодействовать с данными о товарах при помощи пользовательского интерфейса.

\newpage

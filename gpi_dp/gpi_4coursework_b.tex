\documentclass[12pt, a4paper, simple]{eskdtext}

\usepackage{env}
\usepackage{_sty/gpi_lst}
\usepackage{hyperref}

% Код
\def \gpiDocTypeNum {12}
\def \gpiDocVer {00}
\def \gpiCode {\gpiLetterI\gpieLetterII\gpiLetterIII.\gpiStudentGroupName\gpiStudentGroupNum.\gpiStudentCard-0\gpiDocNum~\gpiDocTypeNum~\gpiDocVer}

\def \gpiDocTopic {ТЕКСТ ПРОГРАММЫ}

% Графа 1 (наименование изделия/документа)
\ESKDcolumnI {\ESKDfontIII \gpiTopic \\ \gpiDocTopic}

% Графа 2 (обозначение документа)
\ESKDsignature {\gpiCode}

% Графа 4 (литералы)
\ESKDcolumnIVfI {\gpiLetterI}
\ESKDcolumnIVfII {\gpieLetterII}
\ESKDcolumnIVfIII {\gpiLetterIII}

% Графа 9 (наименование или различительный индекс предприятия) задает команда
\ESKDcolumnIX {\gpiDepartment}

% Графа 11 (фамилии лиц, подписывающих документ) задают команды
\ESKDcolumnXIfI {\gpiStudentSurname}
\ESKDcolumnXIfII {\gpiTeacherSurname}
\ESKDcolumnXIfV {\gpiTeacherSurname}

\begin{document}
    % Титульный лист
    \begin{ESKDtitlePage}
    \begin{flushright}
        \textbf{ПРИЛОЖЕНИЕ Б} \enspace\enspace
    \end{flushright}
    \begin{center}
        \gpiMinEdu \\
        \gpiEdu \\
        \gpiKaf \\
    \end{center}

    \vfill

    \begin{center}
        \gpiTopic \\
    \end{center}

    \vfill

    \begin{center}
        \textbf{\gpiDocTopic} \\
        %(на оптическом носителе DVD-R)
    \end{center}

    \vfill

    \begin{center}
        \gpiCode \\
        Листов \pageref{LastPage} \\
        %Объём XX.XX Мбайт \\
    \end{center}

    \vfill

    \begin{flushright}
        \begin{minipage}[t]{.49\textwidth}
            \begin{minipage}[t]{.75\textwidth}
                \begin{flushright}
                    Руководитель

                    Выполнил

                    Консультант

                    по ЕСПД
                \end{flushright}
            \end{minipage}
        \end{minipage}
        \begin{minipage}[t]{.49\textwidth}
            \begin{flushright}
                \begin{minipage}[t]{.75\textwidth}
                    \gpiTeacherName~\gpiTeacherSurname

                    \gpiStudentName~\gpiStudentSurname

                    \hspace{0pt}

                    \gpiTeacherName~\gpiTeacherSurname

                \end{minipage}
            \end{flushright}
            
        \end{minipage}
    \end{flushright}

    \vfill

    \begin{center}
        \ESKDtheYear
    \end{center}
\end{ESKDtitlePage}

    
    Исходный код на GitHub:
    
    \url{https://github.com/Pavel-Innokentevich-Galanin/gpi_4coursework}
    
    \subsection*{Backend API - Node JS Express}
    \lstinputlisting[]{../gpi_ba/.env.copy}
    \lstinputlisting[]{../gpi_ba/package.json}
    \lstinputlisting[]{../gpi_ba/src/index.js}
    \lstinputlisting[]{../gpi_ba/src/scripts/gpi_function_get_POST.js}
    \lstinputlisting[]{../gpi_ba/src/scripts/gpi_class_Query.js}
    \lstinputlisting[]{../gpi_ba/src/scripts/gpi_class_QueryProducts.js}
    \lstinputlisting[]{../gpi_ba/src/scripts/gpi_class_ObjectProduct.js}
    \lstinputlisting[]{../gpi_ba/src/routes/gpi_router_auth.js}
    \lstinputlisting[]{../gpi_ba/src/routes/gpi_router_products.js}
    \newpage

    % \subsection*{Frontend adminpanel - React JS}

    % \subsection*{Frontend webstore - React JS}
\end{document}

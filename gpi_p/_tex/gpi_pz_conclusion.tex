\newpage

\section*{Заключение}
\addcontentsline{toc}{section}{Заключение}

В ходе выполнения данной курсовой работы усвоили и закрепили знания о работе с клиент-серверной архитектурой:
создание клиенкой части, создание серверной части.

Были разработаны различные виды проектов:

\begin{itemize}
    \item проект сервера, который возвращает JSON данные при URL запросе;
    \item виртуальная машина с баззой данных;
    \item веб-сайт с панелью администратора;
    \item веб-сайт пользователя с корзиной товаров.
\end{itemize}
	
В итоге была разработана программа,
позволяющая добавлять, удалять редактировать товары, которые выводятся на сайт пользователю.

Данная программа может быть полезна различным организациям,
которым необходимо выводить товары на сайт.

Так как большинство СНГ компаний для учёта товаров на складах (занесения записей с накладных) используют
1C Предприятие, то продолжением это проекта можно поставить цель следующую: 
разработать отчёт или функцию в 1C Конфигураторе, который/которая будет сохранять в файл JSON формат.
Раз в день загружать количество товаров на сервер.  

\newpage
